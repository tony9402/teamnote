% Team Note Sample Template
% These codes should be guaranteed, fast enough, short and easy to type.

% \documentclass[8pt, a4paper, oneside, twocolumn, landscape]{extarticle}
\documentclass[8pt, a4paper, oneside, landscape]{extarticle}
% \documentclass[portrait, 8pt, a4paper, oneside, twocolumn]{extarticle} => Warning 보기 싫어서 수정
\usepackage{teamnote}

% Uncomment below to use Korean
\usepackage{kotex}
\usepackage{pygmentize}
\usepackage{multicol}
\usepackage{titlesec}
\usepackage{enumitem}

\teamnote{Hello\, BOJ 2025!}{tony9402}

\titlespacing*{\section}{0pt}{5pt}{2pt}
\titlespacing*{\subsection}{0pt}{5pt}{2pt}

\ShowUsage
\ShowComplexity
\HideAuthor

\begin{document}

\begin{multicols*}{2}

\maketitlepage

% \pagebreak 

\section{Data Structure}


\Algorithm{Dynamic Segment Tree}
{}
{}
{}
{cpp}{tmp/data_structure/dynamic_segment_tree.cpp}
{tony9402}



\Algorithm{Dynamic Segment Tree With Lazy}
{}
{}
{}
{cpp}{tmp/data_structure/dynamic_segment_tree_with_lazy.cpp}
{tony9402}



\Algorithm{Fenwick}
{}
{}
{}
{cpp}{tmp/data_structure/fenwick.cpp}
{tony9402}



\Algorithm{Hld}
{}
{}
{}
{cpp}{tmp/data_structure/hld.cpp}
{tony9402}



\Algorithm{Kdtree}
{}
{}
{}
{cpp}{tmp/data_structure/kdtree.cpp}
{tony9402}



\Algorithm{Lca}
{}
{}
{}
{cpp}{tmp/data_structure/lca.cpp}
{tony9402}



\Algorithm{Pbds}
{}
{}
{}
{cpp}{tmp/data_structure/pbds.cpp}
{tony9402}



\Algorithm{Pst}
{}
{}
{}
{cpp}{tmp/data_structure/pst.cpp}
{tony9402}



\Algorithm{Rope}
{}
{}
{}
{cpp}{tmp/data_structure/rope.cpp}
{tony9402}



\Algorithm{Segment Tree}
{}
{}
{}
{cpp}{tmp/data_structure/segment_tree.cpp}
{tony9402}



\Algorithm{Segment Tree With Lazy}
{}
{}
{}
{cpp}{tmp/data_structure/segment_tree_with_lazy.cpp}
{tony9402}



\Algorithm{Union Find Roll Back}
{}
{}
{}
{cpp}{tmp/data_structure/union_find_roll_back.cpp}
{tony9402}


\section{Graph}


\Algorithm{Dinic}
{}
{}
{}
{cpp}{tmp/graph/flow/dinic.cpp}
{tony9402}



\Algorithm{Mcmf}
{}
{}
{}
{cpp}{tmp/graph/flow/mcmf.cpp}
{tony9402}



\Algorithm{2Sat}
{}
{}
{}
{cpp}{tmp/graph/others/2sat.cpp}
{tony9402}



\Algorithm{Scc}
{}
{}
{}
{cpp}{tmp/graph/others/scc.cpp}
{tony9402}



\Algorithm{Dominator Tree}
{}
{}
{}
{cpp}{tmp/graph/tree/dominator_tree.cpp}
{tony9402}



\Algorithm{Gomory Hu}
{}
{}
{}
{cpp}{tmp/graph/tree/gomory_hu.cpp}
{tony9402}



\Algorithm{Tree Isomorphism}
{}
{}
{}
{cpp}{tmp/graph/tree/tree_isomorphism.cpp}
{tony9402}


\section{Others}


\Algorithm{Fastinput}
{}
{}
{}
{cpp}{tmp/others/fastinput.cpp}
{tony9402}



\Algorithm{Main}
{}
{}
{}
{cpp}{tmp/others/main.cpp}
{tony9402}



\Algorithm{Random}
{}
{}
{}
{cpp}{tmp/others/random.cpp}
{tony9402}


\section{Math}


\Algorithm{Euler Phi}
{}
{}
{}
{cpp}{tmp/math/euler_phi.cpp}
{tony9402}



\Algorithm{Fft}
{}
{}
{}
{cpp}{tmp/math/fft.cpp}
{tony9402}



\Algorithm{Ntt}
{}
{}
{}
{cpp}{tmp/math/ntt.cpp}
{tony9402}


\section{String}


\Algorithm{Aho Corasick}
{}
{}
{}
{cpp}{tmp/string/aho_corasick.cpp}
{tony9402}



\Algorithm{Kmp}
{}
{}
{}
{cpp}{tmp/string/kmp.cpp}
{tony9402}



\Algorithm{Manacher}
{}
{}
{}
{cpp}{tmp/string/manacher.cpp}
{tony9402}



\Algorithm{Z}
{}
{}
{}
{cpp}{tmp/string/z.cpp}
{tony9402}


\section{Geometry}


\Algorithm{Ccw}
{}
{}
{}
{cpp}{tmp/geometry/ccw.cpp}
{tony9402}



\Algorithm{Convex Hull}
{}
{}
{}
{cpp}{tmp/geometry/convex_hull.cpp}
{tony9402}

% 추가해야하는 알고리즘 또는 자료구조 정리
% Centroid
% 스플레이 트리
% 스프라그-그런디 정리 -> 코드보다 사용법?
% LiChao Tree
% 커넥션 DP
% SOS DP
% Convex Hull Trick
% Alien's Trick
% DnC Opt DP

\begin{minted}{cpp}
    < 10^k          number     divisors   2 3 5 71113171923293137
    -------------------------------------------------------------
    1                    6            4   1 1
    2                   60           12   2 1 1
    3                  840           32   3 1 1 1
    4                 7560           64   3 3 1 1
    5                83160          128   3 3 1 1 1
    6               720720          240   4 2 1 1 1 1
    7              8648640          448   6 3 1 1 1 1
    8             73513440          768   5 3 1 1 1 1 1
    9            735134400         1344   6 3 2 1 1 1 1
    10          6983776800         2304   5 3 2 1 1 1 1 1
    11         97772875200         4032   6 3 2 2 1 1 1 1
    12        963761198400         6720   6 4 2 1 1 1 1 1 1
    13       9316358251200        10752   6 3 2 1 1 1 1 1 1 1
    14      97821761637600        17280   5 4 2 2 1 1 1 1 1 1
    15     866421317361600        26880   6 4 2 1 1 1 1 1 1 1 1
    16    8086598962041600        41472   8 3 2 2 1 1 1 1 1 1 1
    17   74801040398884800        64512   6 3 2 2 1 1 1 1 1 1 1 1
    18  897612484786617600       103680   8 4 2 2 1 1 1 1 1 1 1 1
    
    < 10^k    prime   # of prime          < 10^k            prime
    -------------------------------------------------------------
    1             7            4          10           9999999967
    2            97           25          11          99999999977
    3           997          168          12         999999999989
    4          9973         1229          13        9999999999971
    5         99991         9592          14       99999999999973
    6        999983        78498          15      999999999999989
    7       9999991       664579          16     9999999999999937
    8      99999989      5761455          17    99999999999999997
    9     999999937     50847534          18   999999999999999989
\end{minted}

\begin{itemize}[noitemsep]
    
    \item Burnside’s Lemma\\
    - 수식\\
    G=(X,A): 집합X와 액션A로 정의되는 군G에 대해, $\vert A\vert\vert X/A \vert=sum(\vert \text{Fixed points of a}\vert,\text{for all a in A})$\\
    X/A 는 Action으로 서로 변형가능한 X의 원소들을 동치로 묶었을때 동치류(파티션) 집합\\
    - 풀어쓰기\\
    orbit: 그룹에 대해 두 원소 a,b와 액션f에 대해 f(a)=b인거에 간선연결한 컴포넌트(연결집합)\\
    orbit개수 = sum(각 액션 g에 대해 f(x)=x인 x(고정점)개수)/액션개수\\
    - 자유도 치트시트\\
    회전 n개: 회전i의 고정점 자유도=gcd(n,i)\\
    임의뒤집기 n=홀수: n개 원소중심축(자유도 (n+1)/2)\\
    임의뒤집기 n=짝수: n/2개 원소중심축(자유도 n/2+1) + n/2개 원소안지나는축(자유도 n/2)
    
    \item 알고리즘 게임\\
    - Nim Game의 해법(마지막에 가져가는 사람이 승) : XOR = 0이면 후공 승, 0 아니면 선공 승\\
    - Subtraction Game : 한 번에 k 개까지의 돌만 가져갈 수 있는 경우, 각 더미의 돌의 개수를 k + 1로 나눈 나머지를 XOR 합하여 판단한다.\\
    - Index-k Nim : 한 번에 최대 k개의 더미를 골라 각각의 더미에서 아무렇게나 돌을 제거할 수 있을 때, 각 binary digit에 대하여 합을 k + 1로 나눈 나머지를 계산한다. 만약 이 나머지가 모든 digit에 대하여 0이라면 두번째, 하나라도 0이 아니라면 첫번째 플레이어가 승리.\\
    - Misere Nim : 모든 돌 무더기가 1이면 N이 홀수일 때 후공 승, 그렇지 않은 경우 XOR 합 0이면 후공 승
    
    \item Pick’s Theorem\\
    격자점으로 구성된 simple polygon이 주어짐. I 는 polygon 내부의 격자점 수, B 는 polygon 선분 위 격자점 수, A는 polygon의 넓이라고 할 때, 다음과 같은 식이 성립한다. $A=I+B/2-1$
    
    \item 홀의 결혼 정리 : 이분그래프(L-R)에서, 모든 L을 매칭하는 필요충분 조건 = L에서 임의의 부분집합 S를 골랐을 때, 반드시 (S의 크기) $<=$ (S와 연결되어있는 모든 R의 크기)이다.
    
    \item Simpson 공식 (적분) : Simpson 공식, $S_n(f) = \frac{h}{3}[f(x_0)+f(x_n)+ 4\sum f(x_{2i+1}) + 2\sum f(x_{2i})]$\\
    - $M = \max \vert f^4(x) \vert$이라고 하면 오차 범위는 최대 $E_n \leq \frac{M(b-a)}{180}h^4$
    
    \item 브라마굽타 : 원에 내접하는 사각형의 각 선분의 길이가 $a, b, c, d$일 때\\
    사각형의 넓이 $S=\sqrt{(s-a)(s-b)(s-c)(s-d)}$, $s=(a+b+c+d)/2$
    
    \item 브레치나이더 : 임의의 사각형의 각 변의 길이를 $a,b,c,d$라고 하고, 마주보는 두 각의 합을 2로 나눈 값을 $\theta$라 하면, $S=\sqrt{(s-a)(s-b)(s-c)(s-d)-abcd\times cos^2 \theta}$
    
    \item 페르마 포인트 : 삼각형의 세 꼭짓점으로부터 거리의 합이 최소가 되는 점\\
    $2\pi/3$ 보다 큰 각이 있으면 그 점이 페르마 포인트, 그렇지 않으면 각 변마다 정삼각형 그린 다음, 정삼각형의 끝점에서 반대쪽 삼각형의 꼭짓점으로 연결한 선분의 교점\\
    $2\pi/3$ 보다 큰 각이 없으면 거리의 합은 $\sqrt{(a^2 + b^2 + c^2 + 4\sqrt 3 S) / 2}$, $S$는 넓이
    
    \item 오일러 정리: 서로소인 두 정수 $a,n$에 대해 $a^{\phi(n)}\equiv 1 \pmod n$\\
    모든 정수에 대해 $a^n \equiv a^{n-\phi(n)} \pmod n$\\
    $m\geq log_2 n$이면 $a^m\equiv a^{m\%\phi(n)+\phi(n)}\pmod n$
    
    \item $g^0+g^1+g^2+\cdots g^{p-2}\equiv -1 \pmod p$ iff $g=1$, otherwise $0$.
    
    \item if $n \equiv 0 \pmod 2$, then $1^n + 2^n + \cdots + (n-1)^n \equiv 0 \pmod n$
    
    \item Eulerian numbers\\
    Number of permutations $\pi \in S_n$ in which exactly $k$ elements are greater than the previous element. $k$ $j$:s s.t. $\pi(j)>\pi(j+1)$, $k+1$ $j$:s s.t. $\pi(j)\geq j$, $k$ $j$:s s.t. $\pi(j)>j$.\\
    $E(n,k) = (n-k)E(n-1,k-1) + (k+1)E(n-1,k)$\\
    $E(n,0) = E(n,n-1) = 1$\\
    $E(n,k) = \sum_{j=0}^k(-1)^j\binom{n+1}{j}(k+1-j)^n$
    
    \item Pythagorean triple: $a^2+b^2=c^2$이고 서로소인 $(a,b,c)$ 생성\\
    $(a, b, c) = (st, \frac{s^2-t^2}{2}, \frac{s^2+t^2}{2})$, $gcd(s,t)=1, s>t$
    
\end{itemize}

\end{multicols*}

\end{document}
